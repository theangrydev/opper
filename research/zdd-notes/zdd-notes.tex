\documentclass{article}

\usepackage[utf8]{inputenc}
\usepackage[english]{babel}
\usepackage{amsthm}
\usepackage{amsfonts}

\theoremstyle{definition}
\newtheorem{definition}{Definition}[section]

\title{ZDD Notes}
\author{Liam Williams}

\newcommand*\negation[1]{\bar{#1}}
\newcommand*\union[1]{\cup{#1}}

\begin{document}

\section{Definitions}

\begin{definition}
The \textit{Boolean space} $B$ is the set with elements $\{0,1\}$.
\end{definition}

\begin{definition}
A \textit{Boolean variable} is either the constant $0 \in B$ or $1 \in B$.
\end{definition}

\begin{definition}
The \textit{negation} of a Boolean variable $b$ is denoted $\negation{b}$ and is such that: $\negation{0} = 1$ and $\negation{1} = 0$.
\end{definition}

\begin{definition}
A \textit{literal} is a Boolean variable or its negation, e.g. $a$, $\negation{b}$.
\end{definition}

\begin{definition}
A \textit{product}, or \textit{cube} is a Boolean product of literals, e.g. $b\negation{c} \in B^2$.
\end{definition}

\begin{definition}
A \textit{cover} is a set of products, e.g. $\{b\negation{c}, b\} \subset B^2$.
\end{definition}

\begin{definition}
The \textit{cardinality} of a cover $C$ is the number of cubes in the cover. It is denoted $|C|$, e.g. $|\{ab\negation{c}, b\}| = 2$.
\end{definition}

\begin{definition}
A \textit{completely specified Boolean function (CSF)} is a function of the form $f:B^k \rightarrow B$ for some $k \in \mathbb{N}$.
\end{definition}

\begin{definition}
The Boolean function $1$ is the constant boolean function that always maps to $1 \in B$.
\end{definition}

\begin{definition}
A \textit{complement}, or a negative phase, of a cover $C$ is a cover $D$ such that $C \union D$ is a \textit{tautology}, that is, the Boolean function $1$.
\end{definition}


\end{document}